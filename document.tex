\documentclass{article} % This command is used to set the type of document you are working on such as an article, book, or presenation

\usepackage{geometry} % This package allows the editing of the page layout
\usepackage{amsmath}  % This package allows the use of a large range of mathematical formula, commands, and symbols
\usepackage{graphicx}  % This package allows the importing of images
\usepackage{hyperref}
\title{Documentation}
\author{Kai Daniel Gonzalez}

\begin{document}
	\maketitle
	\tableofcontents
	\section{The Start}
	This document documents the document, according
	to documentation when documenting a document, documentation
	should document the properties of a document using that document's documentation, if no documentation exists for that document then refer to documentation or create your own.
	\subsection{Selection}
	
	
	With that in mind, 70 documents were parsed last week, by me, for the documentation. The documentation is important for the documentation of documentation, but that's just being.

	Instead of getting into the logistics and technicalities of the documentation, I'm going to let you figure it out.
	
	\subsection{Calculating Promises}
	
	9 times out of ten means that there's \(9/10\) chances of whatever you're challenging. But if you ever calculated it, it would evaluate to `0.9`, so there's a 0.9 chance of success?
	
	Think about it like this, if it works nine times out of ten, would it work 0.9 times out of ten? no.
	
	\(0.9\) is the chance it has of succeeding. Mathematically, the formula for success according to the SION (Science Institute Of Nowhere):
	
	\begin{quote}
		Logically, success is the reinterpretation of \( \dfrac{x}{y}\), if you have 1/10 chance you need to get the ratio from \emph{0.1}-\emph{10}
	\end{quote}

	In our case, it would be 9 times out of ten chances, which means for 1 in every 9 tries it would fail, since we're getting the ratio as in 0.9 \- 1 is 
  (9 to 10)
	
	\section{Internet Subculture}
	
  There is many internet subcultures, such as furries, cosplayers, Otherkin, and more.

  Personally, I feel that the discrimintation that some
  subcultures get due to the nature of them is absolute trash, 
  and the hate needs to stop (personally)

  Now I wouldn't ever go out of my way to hate on the people
  who hate any internet subculture, because clearly they have their own problems. And
  instead of hating them, we should help our challenged brothers and sisters in understanding
  how us internet users enjoy doing certain things.

  \subsection{Furries}

  Furries are defined as people who enjoy anthromorphicised animals.. Is that how you spell that word?

  Ahh anyway, uhm, furries get a lot of hate due to a few of them being attracted to animals, which
  is known as \href{https://en.wikipedia.org/wiki/Zoophilia}{Zoophilia}.
  A select amount of furries may fall under the category of `Zoophiles' which is not the entire community.
  But sadly, the argument of some rotten apples does not work. People just don't listen,
  you could try to enlighten people but you can't teach anyone who doesn't want to listen.
  
  \section{End}

  This is the end of the EBFI\@. The Enlightenment Book For the Internet.
\end{document}
